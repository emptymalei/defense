\documentclass[12pt, letterpaper]{article}
\usepackage[utf8]{inputenc}

\title{Summary of work since 07/17}
\author{}
\date{\today}

\begin{document}

\maketitle


\section{Neutrino Flavor Conversions and Dispersion Relation}

Izaguirre, Reffelt, and Tamborra published a paper discussing the relations between flavor instabilities and dispersion relation. They considered a two flavor sinario of neutrino oscillations and worked out the dispersion relation between the collective oscillation frequency $\omega$ and wave vector $\vec k$. They identified the possible correlation between instabilities and gaps, that the gaps of $\omega$ in dispersion relations .

We explored this idea and concluded that gaps in dispersion relations are much more complicated than expected. First of all, instabilities do not always correspond to gaps. The simplest example is cases where more than 2 emission beams are used. Even for continuous emissions, we found instabilties happen even without gaps if crossing appears in the spectrum. Finally, we argue that gap should be defined as the gap between dispersion relations and the axis where $\omega=0$ for continuous spectrum.

\section{Neutrino Halo Problem}

In supernova explosions, neutrinos are scattered and producing a neutrino halo. Neutrino self interactions depends on the momenta of interacting neutrinos. The interaction for headon collisions of neutrinos becomes significant in the neutrino halo. Mathematically speaking, the problem is nonlocal boundary value problem.

For simplicity we investigate the scenario that neutrinos are reflected from a surface with some reflection probability. To numerically solve the flavors of neutrinos, we built a relaxition scheme that start from some initial configuration and allow the flavors to relax to an equilibrium state.









\end{document}
